\documentclass[a4paper,11pt]{article}

\begin{document}

\begin{titlepage}
\raggedleft
{\Huge\scshape How To Brick a Microcontroller}
\end{titlepage}

\section{Introduction}

This is an indepth technical adventure into the realm of microcontrollers: what
they are, example use cases, how to program and debug them.

Usually, the goal is \emph{not} to brick microcontrollers and after reading
through this document, the chance of inadvertently doing so should be much
reduced.

Have fun!

\section{Microcontroller Overview}

There are microprocessors, microcontrollers and System on a Chip (SoC) devices
which can seem bewildering at first. How are they different and where do they
overlap?  Roughly, the distinction can be made by what components are included
on the physical chip.

The microprocessor, or Central Processor Unit (CPU), in a desktop or laptop
computer typically requires external IO, ROM and RAM components to form a
complete system: microcontrollers and SoC devices combine these core components
in a single tiny device. Some even include audio and video IO capabilities.

Both bundling, and not bundling, components has disadvantages and advantages:
modular systems notably computers can usually have parts readily upgraded and
replaced whereas integrated devices are a single unit to replace and work with.

There's no need to design or acquire additional circuitry. Development can be
simplified as the system has been designed and supplied by a single
manufacturer. There's no need to add support or detection logic for different
components or configurations as targetting a specific microcontroller
implicitly includes a specific set of components.

\section{Example Use Cases}

Scenarios where using a microcontroller makes sense.

Typical advantages of microcontrollers are low power consumption and compact
size. The chips are usually passively cooled. This makes them appealing for
battery powered devices and, well, small devices as few additional components
are required.

Some microcontrollers are specifically designed to handle temperatures and
operating conditions that computers and other hardware may find extreme -
including the lack of fans which could get clogged with dust.

\section{Programming Microcontrollers}

The indepth content. How to program microcontrollers. From assembly or a higher
level language to power up!

\section{Debugging}

Tips and tricks when things do not work as planned or intended.

\end{document}

